This section introduces a formalization of the \lang{} language. We first introduce the syntax and semantics of the language, and afterwards present the judgment for deriving trust between principals in the language. Finally, this section presents a type system for \lang. 

Throughout this section we will use the term \emph{node} to refer to named principals when talking about the machine on which the named principal is executing code.

\subsection{Syntax}

Figure~\ref{fig:language-syntax} shows the syntax of \lang. The meta-variable $\val$ ranges over terms, which include boolean literals, the unit value, runtime representations of principals, addresses and variables, abstractions $\abs{n}{\lb{p}{\type}}{x}{\expr}$, which are parameterized over the principal name $n$ on which to evaluate the expression $\expr$, and the principal $p$ representing the label protecting the value returned from invoking the abstraction. Finally, the language includes lists using $\nil$ and $\consop$ (pronounced ``cons'') to model \emph{strategies}, which we explain in Section~\ref{sec:semantics}.

Expressions are ranged over by the meta-variable $\expr$ and include terms, applications, projections, elimination of booleans and lists, recursive functions, monadic expressions using $\returnkw$ and $\bindop$, allocation, reading and writing of references. Following LIO \cite{SRMMlio} the language supports operations for labeling and unlabeling expressions, and the operation $\tolabeledkw$ for controlling label-creep \cite{SRMMlio}. The operations $\getlabel$ and $\getclearance$ returns the current label and clearance respectively, and $\labelofkw$ returns the label of a labeled expression. Following \cite{Moore:2016:EAC:2983990.2984021} the scope of delegations is ``fluid'' and the operation $\withscopekw$ creates a new fluid scope. Strategies (introduced later in this section) are introduced using the $\withstrategykw$ operation, and the current strategy can be obtained using $\getstrategy$. Finally, delegations can be added and queried. The shaded regions describe syntax not part of the surface language, but rather constructs used during evaluation.

Finally, types are ranged over by the meta-variable $\type$ and include standard types like the boolean type; the unit type; and function, list and product types. Non-standard types include the type of FLAM principals, the type of labeled values, the type of LIO computations as well as location-aware reference types.

\begin{figure*}
    \centering
    \begin{tabular}{l}
         $\val ::= \true \mid \false \mid () \mid p \mid \addr \mid x \mid \abs{n}{\lb{p}{\type}}{x}{\expr} \mid (\expr, \expr) \mid \cons{\expr}{\expr} \mid \nil \mid \graybox{\lb{p}{\expr}} \mid \graybox{\lio{\expr}}$ \\
         $\begin{aligned}[t]
         \expr &::= \val \mid \app{\expr}{\expr} \mid \proj{i}{\expr} \mid \ifexpr{\expr}{\expr}{\expr} \mid \case{\expr}{\expr}{\expr} \mid \fix{\expr} \mid \return{\expr} \mid \bind{\expr}{\expr} \\ &\quad \mid
         \new{\expr}{\expr} \mid \readref{\expr} \mid \writeref{\expr}{\expr} \mid \Label{\expr}{\expr} \mid \unlabel{\expr} \mid
         \tolabeled{\expr}{\expr} \mid \getlabel \\ &\quad \mid \getclearance \mid \labelof{\expr} \mid \withscope{\expr} \mid \expr \actsfor \expr \mid \withstrategy{\expr}{\expr} \\ &\quad \mid \adddelegate{\expr}{\expr}{\expr} \mid \getstrategy \mid \graybox{\wait{n}[\type]} \mid \graybox{\resetstrategy{\overline{p}}{\expr}} \\ &\quad \mid \graybox{\resetTolabeled{p}{q}{\expr}} \mid \graybox{\resetscope{\scope}{\expr}}
         \end{aligned}$ \\
         $\type ::= \bool \mid \unit \mid \func{\type}{\type} \mid \listtype{\type} \mid \pair{\type}{\type} \mid \labeltype \mid \labeled{\type} \mid \liotype{\type} \mid \reftype[n]{\type}$ \\
    \end{tabular}
    \caption{The \lang{} language}
    \label{fig:language-syntax}
\end{figure*}

\subsection{Semantics}\label{sec:semantics}
The semantics of \lang{} can be split into two judgments: Local reduction rules, which evaluate an expression on a specific node; and global reduction rules, which models remote procedure calls (RPC) and returns. We first present the local reduction rules.

\paragraph{Local semantics}
The semantics of local reduction rules is given by a structured operational semantics with evaluation contexts \cite{Felleisen:1988:TPF:73560.73576}. We elide the definition of the evaluation context, as this is mostly standard.

We further split the local reduction rules into two categories: Pure reduction rules $\steppure{\expr}{\expr'}$ which reduces expressions independent of memories and independent of which principal is evaluating the expression. The pure reduction rules are seen in Figure~\ref{fig:pure-reductions}. Pure reductions include packing terms into monadic contexts, monadically binding terms, recursive applications, projecting pairs, eliminating booleans and lists and obtaining the label of a labeled value.

\begin{figure}
    \centering
    \begin{mathpar}
    \inferrule{}{\steppure{\efill{\return{\val}}} {\efill{\lio{\val}}}}
    \and
    \inferrule{}{\steppure{\efill{\bind{\lio{\val}}{\expr}}}{\efill{\app{\expr}{\val}}}}
    \and
    \inferrule{}{\steppure{\efill{\fix{\expr}}}{\efill{\app{\expr}{(\fix{\expr})}}}}
    \and
    \inferrule{}{\steppure{\efill{\proj{i}{(\val_1, \val_2)}}}{\efill{\val_i}}}
    \and
    \inferrule{}{\steppure{\efill{\ifexpr{b}{\expr_{\true}}{\expr_{\false}}}}{\efill{\expr_b}}}
    \and
    \inferrule{}{\steppure{\efill{\case{\nil}{\expr_1}{\expr_2}}}{\efill{\expr_1}}}
    \and
    \inferrule{}{\steppure{\efill{\case{(\cons{\expr_\hd}{\expr_\tl})}{\expr_1}{\expr_2}}}{ \efill{\app{\app{\expr_2}{\expr_\hd}}{\expr_\tl}}}}
    \and
    \inferrule{}{\steppure{\efill{\labelof{(\lb{p}{\expr})}}}{\efill{p}}}
    \and
    \inferrule{\steppure{\expr}{ \expr'}}{\steppure{\efill{\expr}}{\efill{\expr'}}}
    \end{mathpar}
    \caption{Pure reductions for \lang.}
    \label{fig:pure-reductions}
\end{figure}

The remaining local reductions all depend either on memory or on the identity of the principal which evaluates the expression. We denote these as ``monadic reductions''.

Before introducing the monadic small-step local reduction, we introduce the remaining necessary concepts: First, a store $\store: \Addr \partialto \val$ is a partial mapping from addresses to terms, and we write the empty store as $\varnothing$. A local configuration is a pair $\config{\store}{\overline{\expr}}$ consisting of a store $\store$ and a stack of expressions $\overline{\expr}$. We use stacks of expressions to handle incoming remote procedure calls (RPC) that ``interrupts'' the current computation in order to evaluate the RPC. A global environment $\env: \Nameset \to \sigma$ is a mapping from names to local environments $\sigma$. A local environment $(\lblkw, \scope, \strategy)$ contains the current label $\lblkw$, the set of delegations local to the principal $\scope$ and the current strategy of the principal. We use record notation for these and write $\sigma.\lblkw$, $\sigma.\scope$ and $\sigma.\strategy$ respectively. We let $\emptyenv$ denote the initial global environment satisfying $\emptyenv(n) = (\bot^{\flowsto}, \nil, \nil)$. That is, the initial global environment maps all names to an initial local environment with the lowest principal in the lattice $(\mathcal{P}, \flowsto)$, the empty list of delegations and the empty strategy. While this notation conflicts with the empty store, the context will always allow for disambiguation.

The small-step relation is written $\step{n; \env}{\config{\store}{\expr}}{\config{\store'}{\expr'}}{\sigma}$ and should be read as ``the global environment is $\env$ and machine $n$ performs a single reduction and updates its local environment from $\env_n$ to $\sigma$''.

Figure~\ref{fig:monadic-reductions} in the appendix shows the local reduction rules. Most of the rules verify some trust relationship between principals, written $\actsforquery{n; \env}{p}{q}{\level}$ and can be read as ``node $n$ proves that $p$ acts for $q$ and uses information labeled up to $\level$ in the process''. We defer the discussing this judgment until Section~\ref{subsec:deriving-trust}.

Rule \ruleref{E-Lift-Pure} lifts pure reductions to monadic reductions. Due to the structure of our evaluation contexts ($\ectx \hole ::= \ldots \mid \ectx \hole ; \overline{\expr}$) an expression can reduce at the top of the expression stack only. This is why pure reductions are defined only for single expressions, while the monadic reductions work over lists of expressions. Rules \ruleref{E-Get-Label} and \ruleref{E-Get-Clearance} returns current label or clearance of the principal upon which the expression is being evaluated. Rule \ruleref{E-App} applies a function to an argument and labels the resulting value with the given principal. This choice may seem strange: Why deviate from the traditional semantics of function application? The answer is that with this approach we can combine local function calls and RPC into the same typing rule, and thus simplify the proofs. However, it is straightforward to extend the language with different syntactic constructions for local function application. It would also be straightforward to apply syntactic sugar that unlabels the returned value in case of local function applications.

Rules \ruleref{E-New}, \ruleref{E-Read}, \ruleref{E-Write}, \ruleref{E-Label} and \ruleref{E-Unlabel} are all equivalent to the ones presented in \cite{SRMMlio}, but now also take into account the possible information-flows arising via deriving trust relationships \cite{Arden:2015:FA:2859845.2859998}. Rules \ruleref{E-ToLabeled-1} and \ruleref{E-ToLabeled-2} evaluates an expression $\expr$, but once $\expr$ is evaluated to a term $\val$, the current label is restored to its state before evaluation of $\expr$ began. This operation helps prevent label creep \cite{SRMMlio}, and security guarantees can be reestablished by labeling $\val$. Note that the presentation of this operation is different from \cite{SRMMlio} in that we avoid interleaving small-step and big-step operations, which simplifies the proofs considerably.

Rules \ruleref{E-Acts-For-True} and \ruleref{E-Acts-For-False} queries the trust relationship between two given principals. Rules \ruleref{E-With-Scope} and \ruleref{E-Reset-Scope} introduce and eliminate a fluid scope respectively. Rules \ruleref{E-With-Strategy} introduces a new strategy and evaluates an expression in the lexical scope of the new strategy. Rule \ruleref{E-Reset-Strategy} eliminates the strategy once the expression reduces to a term. Finally, \ruleref{E-Assume} adds a new delegation. This rule uses the voice operator $\voicekw$ defined in Section~\ref{sec:background} to ensure that the computational context has sufficient integrity to delegate trust on behalf of $q$.

Figure~\ref{fig:use-case-for-fluid-scope} demonstrates the usefulness of fluid scoping for delegations. $\mathsf{Alice}$ invokes a function on $\mathsf{Bob}$'s machine and passes a callback on which to invoke data belonging to $\mathsf{Bob}$. But before $\mathsf{Bob}$ invokes the callback he temporarily delegates integrity to $\mathsf{Alice}$. Once $\withscopekw$ evaluates to a term this trust is removed and $\mathsf{Alice}$ can no longer modify data on behalf of $\mathsf{Bob}$.

\begin{figure}
\centering
\begin{lstlisting}
((*@$\lambda^{\mathsf{Alice}}$@*) _ . [...]
  ((*@$\lambda^{\mathsf{Bob}}$@*) f . x <- unlabel lab
             withScope (assume (*@$\integ{\mathsf{Alice}}$@*) (*@$\actsfor$@*) (*@$\integ{\mathsf{Bob}}$@*) @ (*@($\conf{\bot} \conj \integ{\top}$)@*)
                         f x)) aliceCode) ()
\end{lstlisting}
\caption{Bob delegates integrity to Alice and invokes Alice's callback.}
\label{fig:use-case-for-fluid-scope}
\end{figure}

To see how strategies can be used to prevent label creep consider the following scenario: At some point the query $p \actsfor q$, where $p, q \in \Nameset$ is issued. The current set of delegations is $\scope = \left[ \lb{p_1}{x_1}, \dots, \lb{p_n}{x_n} \right]$. As each $\lb{p_i}{x_i}$ is labeled pair it is not possible to inspect these without unlabeling the pair and raise the current label. We now want to unlabel as few of these as possible in order to prove that $p \actsfor q$. Strategies give the programmer first-class control over how this unlabeling of delegations is performed. For instance, if the programmer knows that there will always be a delegation $\lb{r}{p \actsfor q}$ or $\lb{s}{p \actsfor q}$, the strategy $\left[ r, s \right]$ can be set, which firsts checks if a proof can be found using only delegations with a label that flows to $r$. If that fails, delegations with a label that flows to $s$ is checked. In this way the programmer can choose the strategy that minimizes the label creep for each trust query.

\paragraph{Global semantics}
A global configuration is a triple $\gconfig{n}{\env}{S}$ consisting of a node $n \in \Nameset$ denoting which machine is currently reducing an expression, a global environment $\env$, and a mapping $S$ from names to local configurations. Figure~\ref{fig:global-steps} presents the reduction rules for global configurations. Rule \ruleref{G-Step-Local} lifts a local reduction to a global reduction, and rules \ruleref{G-Step-App} and \ruleref{G-Step-Ret} handle remote procedure calls and returns respectively. When node $n$ sends an RPC to node $m$ we will call $n$ the source machine and $m$ the target machine. The global reduction rule is written as $\gconfig{n}{\env}{S} \gstepsto \gconfig{n'}{\env'}{S'}$ and can be read as ``Machine $n$ updates the environment $\env$ to $\env'$, updates the local environments $S$ to $S'$, and transfers control to machine $n'$''. We write the reflexive, transitive closure of $\gstepsto$ as $\gstepstos$.

We now explain how RPC is modeled. First, \ruleref{G-Step-App} will transfer control to the target machine, and the computation will be wrapped in a $\tolabeledkw$ construct at the top of the execution stack on the target machine to prevent the evaluation of the expression from raising the current label. Finally, the evaluation on source machine is suspended. Second, Rule \ruleref{G-Step-Ret} will return control to a suspended source machine when the top of the execution stack on the target machine has reduced to a term.

The final rule, \ruleref{G-Step-Stop} halts the execution when the last machine has reduced its expression to a term. We use the notation $\bullet(S)$ to mean $S = \config{\store}{\bullet}$ for some store $\store$.

Note that the computation is decentralized and multiple machines with different resources stored in their memories can cooperate and invoke each other, but the semantics is still deterministic in that only one expression will be reducible at any point. This excludes internal timing leaks and other attacks normally found in concurrent systems \cite{Smith:1998:SIF:268946.268975, Muller:2012:TPS:2384616.2384621}, while still allowing multiple nodes to share computation.

\begin{figure*}
\centering
\begin{mathpar}
\inferrule[G-Step-Local]{\step{n; \env}{S_n}{S'}{\sigma}}{\gconfig{n}{\env}{S} \gstepsto \gconfig{n}{\extend{\env}{n}{\sigma}}{\extend{S}{n}{S'}}}
\and
\inferrule[G-Step-App]{\level_n = \env_n.\lblkw \and \level_m = \env_m.\lblkw \and S_m = \config{\store_m}{\overline{\expr_m}} \and \nopostflowstoquery{m; \env}{\level_n \join \level_m}{m} \\\\ S_n = \config{\store_n}{\efill{\app{(\abs{m}{\lb{p}{\type}}{x}{\expr_n})}{\expr_n'}}} \and S_n' = \config{\store_n}{\efill{\wait{m}[\type]}} \\\\ S_m' = \config{\store_m}{(\resetTolabeled{\level_m}{p}{(\expr_n[\expr_n' / x])}) ; \overline{\expr_m}}}{\gconfig{n}{\env}{S} \gstepsto \gconfig{m}{\extend{\env}{m}{\extend{\env_m}{\lblkw}{\level_n \join \level_m}} }{\extends{S}{n \mapsto S_n', m \mapsto S_m'}}}
\and
\inferrule[G-Step-Ret]{S_n = \config{\store_n}{\efill{\wait{m}[\type]}} \and S_m = \config{\store_m}{\val ; \overline{\expr_m}} }{\gconfig{m}{\env}{S} \gstepsto \gconfig{n}{\env}{\extends{S}{n \mapsto \config{\store_n}{\efill{\val}}, m \mapsto \config{\store_m}{\overline{\expr_m}}}}}
\and
\inferrule[G-Step-Stop]{\forall n \in \Nameset \setminus \{m\}~.~ \termsym(S_n) \and S_m = \config{\store_m}{\val ; \termsym}}{\gconfig{m}{\env}{S} \gstepsto[][] \gconfig{m}{\env}{\extend{S}{m}{\config{\store_m}{\termsym}}}}
\end{mathpar}
\caption{Semantics of global steps}
\label{fig:global-steps}
\end{figure*}

\subsection{Deriving trust relationship in \lang}\label{subsec:deriving-trust}
\begin{figure}
    \centering
    \begin{mathpar}
    \inferrule[Bot]{}{\actsforquery{C}{p}{\bot}{\bot^{\flowsto}}}
    \and
    \inferrule[Top]{}{\actsforquery{C}{\top}{p}{\bot^{\flowsto}}}
    \and
    \inferrule[Refl]{}{\actsforquery{C}{p}{p}{\bot^{\flowsto}}}
    \and
    \inferrule[Hyp]{(p \actsfor q) \in \Assumps}{\actsforquery{C}{p}{q}{\bot^{\flowsto}}}
    \and
    \inferrule[Proj]{\actsforquery{C}{p}{q}{\level}}{\actsforquery{C}{\authproj{p}}{\authproj{q}}{\level}}
    \and
    \inferrule[ProjR]{}{\actsforquery{C}{p}{\authproj{p}}{\bot^{\flowsto}}}
    \and
    \inferrule[Own-1]{\actsforquery{C}{o}{o'}{\level_1} \\\\ \actsforquery{C}{p}{p'}{\level_2}}{\actsforquery{C}{\owner{o}{p}}{\owner{o'}{p'}}{\level_1 \join \level_2}}
    \and
    \inferrule[Own-2]{\actsforquery{C}{o}{o'}{\level_1} \\\\ \actsforquery{C}{p}{\owner{o'}{p'}}{\level_2}}{\actsforquery{C}{\owner{o}{p}}{\owner{o'}{p'}}{\level_1 \join \level_2}}
    \and
    \inferrule[Conj-L]{j \in \{ 1, 2 \} \\\\ \actsforquery{C}{p_j}{p}{\level}}{\actsforquery{C}{p_1 \conj p_2}{p}{\level}}
    \and
    \inferrule[Conj-R]{\actsforquery{C}{p}{p_1}{\level_1} \\\\\ \actsforquery{C}{p}{p_2}{\level_2}}{\actsforquery{C}{p}{p_1 \conj p_2}{\level_1 \join \level_2}}
    \and
    \inferrule[Disj-L]{\actsforquery{C}{p_1}{p}{\level_1} \\\\\ \actsforquery{C}{p_2}{p}{\level_2}}{\actsforquery{C}{p_1 \disj p_2}{p}{\level_1 \join \level_2}}
    \and
    \inferrule[Disj-R]{j \in \{ 1, 2 \} \\\\ \actsforquery{C}{p}{p_j}{\level}}{\actsforquery{C}{p}{p_1 \disj p_2}{\level}}
    \and
    \inferrule[Trans]{\actsforquery{C}{p}{q}{\level_1} \\\\ \actsforquery{C}{q}{r}{\level_2}}{\actsforquery{C}{p}{r}{\level_1 \join \level_2}}
    \and
    \inferrule[Del]{\mathsf{ss} = \env_n . \strategy \\\\ \actsforquerysmany{C}{\mathsf{ss}}{p}{q}{\level}}{\actsforquery{C}{p}{q}{\level}}
    \and
    \inferrule[Fwd]{\flowstoquery{C}{\env_n . \lblkw}{m}{\level_1} \\\\ \actsforquery{\Assumps; m; \extend{\env}{m}{\extend{\env_m}{\strategy}{\env_n.\strategy}}}{p}{q}{\level_2} }{\actsforquery{C}{p}{q}{\level_1 \join \level_2}}
    \end{mathpar}
    \caption{Acts for judgment of \lang. The meta-variable $C$ abbreviates $\Assumps; n; \env$}
    \label{fig:act-for-judgment}
\end{figure}

\lang{} allows, in the style of FLAM \cite{Arden:2015:FA:2859845.2859998} the trust relationship between principals to be changed dynamically throughout the evaluation of a program. This means that checking if some principal $p$ currently trusts some other principal $q$ is non-trivial and cannot be done statically. Thus it is important to ensure that no sensitive information is leaked to an attacker when checking such trust relationships. We show how the ideas in \cite{Arden:2015:FA:2859845.2859998} can be incorporated into the floating-label model of LIO. Figure~\ref{fig:act-for-judgment} shows how the judgment $\actsforquery{\Assumps; n; \env}{p}{q}{\level}$ derives trust relationships between principals. Intuitively $\actsforquery{\Assumps; n; \env}{p}{q}{\level}$ means that node $n$ has to raise its current floating label to be at least $\level$ in order to use the information that $q$ trusts $p$. We write $\actsforquery{n; \env}{p}{q}{\level}$ to mean $\actsforquery{\varnothing; n; \env}{p}{q}{\level}$.

Many rules translate directly from \cite{Arden:2015:FA:2859845.2859998}, but differ in how delegations are handled and how remote nodes are used to derive trust relationships. This discrepancy arises because, in FLAM the upper bound on the label of usable delegations is given as ``input'' to the judgment, while in LIO the upper bound is part of the ``output'' of the judgment. Rule \ruleref{Bot} says that the element $\bot$ trusts any principal, and that deriving this trust requires information up to level $\bot^{\flowsto}$. Dually, \ruleref{Top} states that any principal trusts $\top$, and \ruleref{Refl} states that any element trusts itself. Rule \ruleref{Hyp} states that any hypothesis can be used to derive trust without raising the current label any further. We will see that this is secure by relating it to a form of checked endorsement \cite{Chong:2007:SWA:1294261.1294265}. Rule \ruleref{Proj} expresses that applying projections preserve trust between principals and \ruleref{ProjR} states that a principal is at least as trusted as the projected principal. Rules \ruleref{Own-1} and \ruleref{Own-2} derives trust between ownership projections. First, \ruleref{Own-1} shows that trust between principals imply trust between owned principals, and \ruleref{Own-2} states that, if an ownership projection $\owner{o'}{p'}$ trusts a principal $p$ then another ownership principal $\owner{o}{p}$ also trusts $\owner{o'}{p'}$, but only if the the owner $o'$ trusts $o$. Rules \ruleref{Conj-L}, \ruleref{Conj-R}, \ruleref{Disj-L} and \ruleref{Disj-R} correspond to the usual lattice rules for joins and meets, and \ruleref{Trans} ensures that the trust relation is transitive. Rule \ruleref{Fwd} expresses how a node $n$ can query another node $m$ for a trust relationship, but only if $n$ allows the information that caused $n$ to contact $m$ to be learned by $m$. Furthermore, the searching strategy used by $n$ should also be used by $m$.

Finally, rule \ruleref{Del} expresses how delegations can be used to derive trust. The judgment uses the search strategy to control how the information-flow lattice should be searched. This is shown in Figure~\ref{fig:act-for-judgment-del}, where the first element $s$ in the strategy that successfully derives the trust relationship terminates the proof search. We will denote the principals in the current strategy as \emph{strategy principals}. Also note that this is deterministic in the sense that the strategy principals is searched from beginning to end until one succeeds. The rules \ruleref{Del-1} and \ruleref{Del-2} uses an interpretation function $\llbracket \cdot \rrbracket_s(\Assumps, n, \env)$ to compute the set of delegations protected by a label at most $s$ for a node $n$ given assumptions $\Assumps$ and global environment $\env$. This function is defined as
\begin{align}\label{eq:interpret-del}
&\llbracket (\lb{\level}{p \actsfor q}) :: \scope \rrbracket_s(\Assumps, n, \env) = \\ \nonumber &\qquad  \begin{cases}
\{ p \actsfor q \} \cup \scope' & \text{ if } \flowstoquery{\Assumps, p \actsfor q; n; \env'}{\level}{s}{\level'} \\ &\qquad \wedge\ \nopostflowstoquery{\Assumps, p \actsfor q}{\level'}{s} \\
\scope' & \text{ otherwise}
\end{cases}
\\ \nonumber
&\llbracket \nil \rrbracket_s(\Assumps, n, \env) = \varnothing
\end{align}
where $\scope' = \llbracket \scope \rrbracket_s((\Assumps, p \actsfor q), n, \env')$ and $\env' = \env \setminus_n (\lb{\level}{p \actsfor q})$\footnote{We use the notation $\env \setminus_n (\lb{\level}{p \actsfor q})$ to mean the environment $\extend{\env}{n}{\extend{\env_n}{\scope}{\env_n.\scope \setminus (\lb{\level}{p \actsfor q})}}$.}. Intuitively, this function unlabels the delegations that the node $n$ is allowed to observe. That is, if the label on the delegation flows to the current strategy principal. Note the recursive problem: In order to check if $n$ is allowed to observe the delegation $\lb{\level}{p \actsfor q}$ using a strategy principal $s$ it should be checked if $\level \flowsto s : \level'$ for some $\level'$, which then requires information up to level $\level'$ in the lattice $(\mathcal{P}, \flowsto)$, so we need to check that $\level' \flowsto s : \level''$ for some $\level''$, etc. At some point this search would have to terminate with some label that is known to flow to $s$ statically. We increase the permissiveness of this while ensuring a strong sense of security by \emph{allowing $n$ the use of the delegation $p \actsfor q$ when checking if $p \actsfor q$ is allowed to be observed by $n$}. This is a form of checked endorsement \cite{DBLP:journals/corr/abs-1107-5594} which was also noted to be a useful extension to the Jif programming language \cite{Chong:2007:SWA:1294261.1294265}. The judgment $\nopostactsforquery{\Assumps}{p}{q}$ is a judgment checking if $q$ trusts $p$ under the assumptions $\Assumps$. We omit the definition of this judgment, but the full definition can be found in \cite{techreport}.

To see the effect of this style of reasoning, assume $\Nameset = \{n\}$ and consider the query $\flowstoquery{n; \env}{p}{q}{q}$ where $\env_n = (\bot^{\flowsto}, \scope, \cons{q}{\nil})$ and $\scope = \left[\lb{p}{\conf{q} \conj \integ{p} \actsfor \conf{p} \conj \integ{q}} \right]$. This query is equivalent to $\actsforquery{n; \env}{\conf{q} \conj \integ{p}}{\conf{p} \conj \integ{q}}{q}$. Applying \ruleref{Del} and \ruleref{Del-1} the goal reduces to proving $(p \flowsto q) \in \llbracket \scope \rrbracket(\varnothing, n, \env)$. Computing $\llbracket \scope \rrbracket(\varnothing, n, \env)$ we get
\begin{align*}
\llbracket \scope \rrbracket(\varnothing, n, \env) =
\begin{cases}
\{p \flowsto q \} &\text{if } \flowstoquery{\{p \flowsto q\}; n; \emptyenv}{p}{q}{\bot^{\flowsto}} \\
&\quad \wedge \, \nopostflowstoquery{\{p \flowsto q\}; n; \emptyenv}{\bot^{\flowsto}}{q} \\
\varnothing &\text{otherwise}
\end{cases}
\end{align*}
noting that $\env \setminus_{n} (\lb{q}{p \flowsto q}) = \emptyenv$. So $\llbracket \scope \rrbracket(\varnothing, n, \env) = \{p \flowsto q \}$ and thus $\flowstoquery{n; \env}{p}{q}{q}$ holds.
If the condition $\flowstoquery{\{p \flowsto q\}; n; \emptyenv}{p}{q}{\bot^{\flowsto}}$ was changed to the naive condition $\flowstoquery{\varnothing; n; \env}{p}{q}{\bot^{\flowsto}}$, that is, no assumptions was added when checking that the label on the delegation $\lb{p}{\conf{q} \conj \integ{p} \actsfor \conf{p} \conj \integ{q}}$ flows to the strategy principal $q$, this trust relationship could not be proven in any finite derivation.

\paragraph{Non-structurally smaller recursive trust checks}
Figure~\ref{fig:act-for-judgment} and Equation~\eqref{eq:interpret-del} shows that often, to prove a trust relationship between principals, it is necessary to check trust between non-structurally smaller principals. This has implications not just for proofs about trust derivations, but also for practical implementations of the judgment. From a proof point of view, it means that it is not possible to appeal to an induction hypothesis, since the principals involved might not necessarily be structurally smaller. From the point of view of an implementation, it means that is is non-trivial to guarantee termination when checking trust relationships. To solve these issues we use a \emph{step-indexed} \cite{Appel:2001:IMR:504709.504712} judgment $\actsforquerycount{\Assumps; n; \env}{p}{q}{\level}{k}$ stating that $q$ trusts $p$ could be proved by a proof of depth at most $k$. Induction in $k$ then suffices to prove the necessary results. This technique is sound as any successful proof of $\actsforquery{\Assumps; n; \env}{p}{q}{\level}$ must have some bounded depth.

We will omit such step-indexes in the presentation of the rules, but the technical report \cite{techreport} contains all the details.

\begin{figure}
    \centering
    \begin{mathpar}
    \inferrule[Del-1]{\scope = \env_n.\scope \\\\ (p \actsfor q) \in \llbracket \scope \rrbracket_s(\Assumps, n, \env)}{\actsforquerysmany{\Assumps; n; \env}{\mathsf{s}::\mathsf{ss}}{p}{q}{s}}
    \and
    \inferrule[Del-2]{\scope = \env_n.\scope \\\\ (p \actsfor q) \notin \llbracket \scope \rrbracket_s(\Assumps, n, \env) \\\\ \actsforquerysmany{\Assumps; n; \env}{\mathsf{ss}}{p}{q}{\level}}{\actsforquerysmany{\Assumps; n; \env}{\cons{\mathsf{s}}{\mathsf{ss}}}{p}{q}{\level \join s}}
    \end{mathpar}
    \caption{Auxiliary judgment for using delegations when proving an acts for query in \lang.}
    \label{fig:act-for-judgment-del}
\end{figure}

\subsection{A type system for \lang.}
Since \lang{} controls information-flows via dynamic checks, the type system for \lang{} is straightforward. We write $\hastype{n; \tenv}{\expr}{\type}$ when expression $\expr$ can be given type $\type$ in a global type environment $\tenv: \Nameset \to (\Var \to \type)$ on machine $n$. Excerpt of the judgment is shown in Figure~\ref{fig:type-judgment}. Rule \ruleref{T-Principal} states that principals have type $\labeltype$, and \ruleref{T-Var} states that variable types are given by the global type environment $\tenv$. Rule \ruleref{T-Ref} simplifies the statement of Theorem~\ref{thm:preservation} by having the typing environment denote the type of both variables and addresses. We say an address $\addr$ belongs to node $n$ if $\hastype{m; \tenv}{\addr}{\reftype[n]{\type}}$ for some $m$. Rule \ruleref{T-Abs} states that an abstraction has a function type and that, when checking the type of the body, the typing environment for node $m$ is used, where $m$ is the target machine. Rule \ruleref{T-Lab} states that labeled expressions have labeled types. Rules \ruleref{T-Lio}, \ruleref{T-Return} and \ruleref{T-Bind} are standard typing rules for monadic expressions. Rules \ruleref{T-New} states that, when a reference is allocated on a node $n$ the type of the address returned will belong to $n$. Rule \ruleref{T-Read} states that a reference can only be read on a machine on which the address belongs. Finally, \ruleref{T-Wait} states that the type attached to a waiting expression is the type of the expression. The rules \ruleref{T-Wait} and \ruleref{T-Abs} are closely connected in that an RPC will evaluate an abstraction to a waiting expression, meaning that the typing rules for these expressions should match.

\paragraph{Store typing}
Given a global typing environment $\tenv$ we write $\wf{m; \tenv}{\store}$ if, for all $\addr$ such that $\store(\addr) = \lb{p}{\expr}$ and $\tenv_m(\addr) = \reftype[n]{\type}$ it holds that $\hastype{n; \tenv}{\expr}{\type}$. We write $\hastype{n; \tenv}{\config{\store}{\overline{\expr}}}{\overline{\type}}$ if $\wf{n; \tenv}{\store}$ and $\hastype{n; \tenv}{\expr_i}{\type_i}$ for $i = 1, \dots, n$ and $\overline{\expr} = \expr_1 \cdots \expr_n$ and $\overline{\type} = \type_1 \dots \type_n$. We lift this definition to global configurations and write $\hastype[m]{\tenv}{\gconfig{n}{\env}{S}}{\overline{\type}}$ if for all $n' \in \Nameset$ there exists a type $\overline{\type}'$ such that $\hastype{n'; \tenv}{S_{n'}}{\overline{\type}'}$, and furthermore, when $n' = m$ we have $\overline{\type}' = \overline{\type}$.

\paragraph{Preservation}
As the program allocates references during evaluation, the typing environment needs to incorporate these addresses accordingly. Thus Theorem~\ref{thm:preservation} shows that, if a configuration is well-typed and takes a single step, there will exist a global typing environment, which is an extension of the old typing environment, such that the new configuration is well-typed.

\begin{theorem}[Preservation]\label{thm:preservation}
If $\hastype[m]{\tenv}{\gconfig{n}{\env}{S}}{\overline{\type}}$ and $\gconfig{n}{\env}{S} \gstepsto \gconfig{n'}{\env'}{S'}$ then there exists $\tenv' \supseteq \tenv$ and $\overline{\type}$ such that either $\overline{\type}' \leq \overline{\type}$ or $\overline{\type} \leq \overline{\type}'$ and $\hastype[m]{\tenv'}{\gconfig{n'}{\env'}{S'}}{\overline{\type}'}$.
\end{theorem}

We write $\overline{\type_1} \leq \overline{\type_2}$ to mean that $\overline{\type_1}$ is a \emph{prefix} of $\overline{\type_2}$. So Theorem~\ref{thm:preservation} states that the global reduction semantics steps to a global configuration which type is either a prefix or a suffix of the original type. The statement in the technical report~\cite{techreport} is stronger: It proves that $\overline{\type}' \leq \overline{\type}$ when $\gconfig{n}{\env}{S} \gstepsto \gconfig{n'}{\env'}{S'}$ is an application of \ruleref{G-Step-Ret} and $\overline{\type} \leq \overline{\type}'$ when $\gconfig{n}{\env}{S} \gstepsto \gconfig{n'}{\env'}{S'}$ is an application of \ruleref{G-Step-App}, and $\overline{\type} = \overline{\type}'$ otherwise.

\begin{figure}
\centering
\begin{mathpar}
\inferrule[T-Principal]{~}{\hastype{n; \tenv}{p}{\labeltype}}
\and
\inferrule[T-Var]{\tenv_n(x) = \type}{\hastype{n; \tenv}{x}{\type}}
\and
\inferrule[T-Ref]{\tenv_n(\addr) = \reftype[m]{\type}}{\hastype{n; \tenv}{\addr}{\reftype[m]{\type}}}
\and
\inferrule[T-Abs]{\hastype{m; \tenv, x : \type_1}{\expr}{\liotype{\type_2}} \\\\ \type = \func{\type_1}{\liotype{(\labeled{\type_2})}}}{\hastype{n; \tenv}{\abs{p}{\lb{\type_2}{m}}{x}{\expr}}{\type}}
\and
\inferrule[T-Lab]{\hastype{n; \tenv}{\expr_2}{\type} \\\\ \hastype{n; \tenv}{\expr_1}{\labeltype}}{\hastype{n; \tenv}{\lb{\expr_1}{\expr_2}}{\labeled{\type}}}
\and
\inferrule[T-Lio]{\hastype{n; \tenv}{\expr}{\type}}{\hastype{n; \tenv}{\lio{\expr}}{\liotype{\type}}}
\and
\inferrule[T-Return]{\hastype{n; \tenv}{\expr}{\type}}{\hastype{n; \tenv}{\return{\expr}}{\liotype{\type}}}
\and
\inferrule[T-Bind]{\hastype{n; \tenv}{\expr_1}{\liotype{\type_1}} \\\\ \hastype{n; \tenv}{\expr_2}{\func{\type_1}{\liotype{\type_2}}}}{\hastype{n; \tenv}{\bind{\expr_1}{\expr_2}}{\liotype{\type_2}}}
\and
\inferrule[T-New]{\hastype{n; \tenv}{\expr_1}{\labeltype} \\\\ \hastype{n; \tenv}{\expr_2}{\type} \\\\ \type' = \liotype{(\reftype[n]{\type})}}{\hastype{n; \tenv}{\new{\expr_1}{\expr_2}}{\type'}}
\and
\inferrule[T-Read]{\hastype{n; \tenv}{\expr}{\reftype[n]{\type}}}{\hastype{n; \tenv}{\readref{\expr}}{\liotype{\type}}}
\and
\inferrule[T-Wait]{\type' = \liotype{(\labeled{\type})}}{\hastype{n; \tenv}{\wait{m}[\type]}{\type'}}
\end{mathpar}
\caption{Typing judgment for \lang.}
\label{fig:type-judgment}
\end{figure}