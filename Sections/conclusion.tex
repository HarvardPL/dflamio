This paper demonstrates the usefulness of the FLAM authorization logic for a language with coarse-grained dynamic information-flow control, in the style of the floating label model of LIO. The paper shows that the two systems can be combined to obtain a provably strong noninterference result for a language with distributed computation and decentralized trust. The language has been implemented as a monadic library in Haskell, and the usability of the system has been validated via three use cases involving secure, distributed access to shared resources.