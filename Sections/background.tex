Before discussing how FLAM and LIO fit together, we briefly introduce each system separately. We first highlight the important parts of FLAM necessary for this work, and details can be found in \cite{Arden:2015:FA:2859845.2859998}. Similarly, we highlight the important parts of LIO, and additional details can be found in \cite{SRMMlio}.

\subsection{The FLAM principal lattice}
Figure~\ref{fig:flam-syntax} describes the syntax of FLAM principals. The grammar is parametric in a set $\Nameset$ of names representing principals like $\mathsf{Alice}$ and $\mathsf{Bob}$. Given a principal $p$ FLAM gives the ability to talk about the confidentiality and integrity of $p$ using \emph{basis projections} $\conf{p}$ and $\integ{p}$ respectively. The principal $\conf{p}$ represents the authority to learn anything that $p$ can learn, and $\integ{p}$ represents the authority to modify anything that $p$ can modify. Given principals $p$ and $q$ FLAM can also represent the authority of \emph{both} $p$ and $q$ as $p \conj q$ or the authority of \emph{either} $p$ or $q$ as $p \disj q$. This forms a lattice $(\mathcal{P}, \actsfor)$ with the partial order $\actsfor$ (pronounced ``acts for''\footnote{The relation $p \actsfor q$ can also be interpreted as ``$q$ trusts $p$''.}), where $\bot$ and $\top$ represents the least and most trusted principals respectively, and where the elements are given by the grammar in Figure~\ref{fig:flam-syntax}. We denote the set of principals by $\mathcal{P}$.\footnote{Formally, the elements in the lattice is the equivalence class of $\mathcal{P}$ modulo the relation $\equiv$ where $a \equiv b \iff a \actsfor b$ and $b \actsfor a$.}.

Besides basis projections $\rightarrow$ and $\leftarrow$ FLAM also defines \emph{ownership projections} $\owner{p}{q}$ representing the principal $q$, but where the owner $p$ controls which principals $\owner{p}{q}$ should trust. We will use ownership extensively in the example presented in Section~\ref{subsec:agents}.

\begin{figure}
    \centering
    \begin{tabular}{ll}
    $n \in \Nameset$ \\
    $p ::= \bot \mid \top \mid \name \mid p \conj p \mid p \disj p \mid \conf{p} \mid \integ{p} \mid \owner{p}{p}$
    \end{tabular}
    \caption{Syntax of FLAM}
    \label{fig:flam-syntax}
\end{figure}


\paragraph{An information-flow ordering}
An important distinction between FLAM and other authorization models is that FLAM unifies trust and information-flow into a single concept. Specifically, FLAM defines the operations
\begin{align*}
p \flowsto q &\circeq \conf{q} \conj \integ{p} \actsfor \conf{p} \conj \integ{q}\\
p \join q &\circeq \conf{(p \conj q)} \conj \integ{(p \disj q)} \\
p \meet q &\circeq \conf{(p \disj q)} \conj \integ{(p \conj q)}
\end{align*}
That is, $p \flowsto q$ (pronounced $p$ ´´flows to'' $q$) if $q$ acts for the confidentiality of $p$, and $p$ acts for the integrity of $q$. The ``join'' of $p$ and $q$, written $p \join q$ is defined as the principal with the authority of both $p$'s and $q$'s confidentiality, and the authority of either $p$'s or $q$'s integrity. The ``meet'' of $p$ and $q$, written $p \meet q$ is defined dually as the confidentiality of either $p$ or $q$, and the integrity of both $p$ and $q$.
This forms a lattice $(\mathcal{P}, \flowsto)$ with the partial order $\flowsto$, where the bottom element $\bot^{\flowsto} \circeq \conf{\bot} \conj \integ{\top}$ represents the least confidential and most trusted principal, and the top element $\top^{\flowsto} \circeq \conf{\top} \conj \integ{\bot}$ represents the most confidential and least trusted principal\footnote{Like for the lattice representing trust, the elements in the information-flow lattice is equivalence classes of $\mathcal{P}$ modulo the relation $\equiv$ defined as $a \equiv b \iff a \flowsto b \wedge b \flowsto a$.}.

\paragraph{Voice of a principal}
Finally, FLAM defines the \emph{voice} of a principal $p$, denoted $\voice{p}$, as the minimum integrity needed to influence the flow of information labeled $p$. Formally, the voice operation is defined for principals in normal form\footnote{In \cite{Arden:2015:FA:2859845.2859998} the authors show how any FLAM principal $p$ can be factored into a conjunction of a confidentiality projection $\conf{q}$ and an integrity projection $\integ{r}$. This is called the normal form of $p$.} $\conf{p} \conj \integ{q}$ as $\voice{\conf{p} \conj \integ{q}} = \integ{p} \conj \integ{q}$.

\subsection{Coarse-grained information flow using LIO.}
LIO \cite{SRMMlio} is a Haskell library for dynamic information-flow control. LIO is parametric in the label model and takes a coarse-grained approach to information-flow using a \emph{floating label model}: Instead of attaching a label to each value in the program, the \emph{computational context} is protected with a label called the current label. Throughout the execution of the program this label will ``climb'' up the information-flow lattice until it is required to climb above the \emph{clearance level} of the program, at which the program aborts. The type of LIO computations form a monad \cite{Wadler:1995:MFP:647698.734146}, which makes programming with LIO convenient in Haskell. Specifically, the operations $\return{\expr}$ embeds a pure expression $\expr$ into the LIO computational context, and the operation $\bind{\expr_1}{\expr_2}$ (pronounced ``bind'') chains together monadic LIO operations $\expr_1$ and $\expr_2$.

\paragraph{Labeled values}
As LIO protects every value in the computational context by a single label $\level_{\mathsf{cur}}$, it must provide a way to protect some values with a label higher than $\level_{\mathsf{cur}}$ (ie., a value representing a password should not be protected merely by $\level_{\mathsf{cur}}$ since it should not be observable to an attacker). To do this, LIO provides two useful operations:
\begin{minted}{haskell}
label :: Label l => a -> l -> LIO l (Labeled l a)
unlabel :: Label l => Labeled l a -> LIO l a
\end{minted}
Here, \mintinline{haskell}{Label l} is a typeclass constraint specifying that the type \mintinline{haskell}{l} must be an instance of the \mintinline{haskell}{Label} typeclass, meaning that \mintinline{haskell}{l} must have operations $\join$, $\meet$ and $\flowsto$. The operation \mintinline{haskell}{label} takes an expression $\expr$ and a label $\level$, and protects $\expr$ with the label $\level$. This labeled value can then be passed around without raising the label of the computational context. When the value is needed, the operation \mintinline{haskell}{unlabel} must be invoked, which gets back the value and raises the current label $\level_{\mathsf{cur}}$ to $\level_{\mathsf{cur}} \join \level$, while checking that the new current label flows to the clearance label $\level_{\mathsf{clr}}$ of the application.

\paragraph{Preventing label creep}
As the program executes and performs \mintinline{haskell}{unlabel} operations, the current label will continue to rise. For instance, imagine adding two labeled numbers together:
\begin{minted}{haskell}
add :: (Num a, Label l) => Labeled l a ->
         Labeled l a -> LIO l a
add lx ly = unlabel lx >>= \x ->
            unlabel ly >>= \y ->
            return (x + y)
\end{minted}
Evaluating \mintinline{haskell}{add x y} will raise the current label $\level_{\mathsf{cur}}$ to $\level_{\mathsf{cur}} \join \level_x \join \level_y$, where $\level_x$ and $\level_y$ represent the labels on the values of $x$ and $y$ respectively. To avoid this \emph{label creep} LIO introduces the following operation:
\begin{minted}{haskell}
toLabeled :: Label l => l -> LIO l ->
               LIO l (Labeled l a)
\end{minted}
The expression \mintinline{haskell}{toLabeled l m} evaluates \texttt{m} and resets the current label to its state before the evaluation of \texttt{m}. To remain secure, the result is labeled with the label \texttt{l} and LIO checks that the evaluation of \texttt{m} never raised the current label above \texttt{l}.