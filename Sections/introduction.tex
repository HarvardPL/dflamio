Most modern systems of today require some form of authorization mechanism to control access to data in various forms. Such authorization mechanisms tend to be complex and hard to get right, even though the correctness of such components is vital for the security of the system and its users. The behavior of such systems is often very dynamic, with access control constantly changing on a per-user basis. Thus, the authorization mechanisms used in such systems also need to be very dynamic and able to securely control runtime changes to user privileges.
This is especially true for distributed systems. In this case, machines located geographically far away from each other, and thus not always up-to-date on the latest trust relationships between users or between users and the system, need to agree on whether to allow a user access to new information or allow a user to modify already existing information in the system.

A common correctness guarantee for the security of a system is noninterference \cite{Cite appropiate NI literature}. Noninterference states, that an attacker only learns what a given policy is allowing them to learn, or dually that an attacker only influences the part of the system that the policy is allowing them to influence. Such strong guarantees is rarely possible in real systems, where release of confidential information can be necessary. For instance, when a user inputs an invalid username and password pair, the system must respond with an appropriate message, allowing an attacker to learn that such a user does not exist in the system. In such cases a \emph{declassification} of confidential information is necessary. Dually, a system trusted will sometimes need to allow untrusted information to influence the result of a trusted computation. For instance during an auction, where untrusted bidders each submit an untrusted bid and the system determines the winner based on the untrusted bid. This is an example of an \emph{endorsement}. The general term for such examples is information \emph{downgrading}\footnote{The ``down'' refers to the direction of flow in the information-flow lattice.}.

