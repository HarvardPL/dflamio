In this section, we define the attacker model and show the security guarantees given by \lang. Specifically, we show that \lang{} executions satisfy termination-insensitive noninterference (TINI) \cite{4223226}. Formally, an attacker is some principal $\Attacker$. Note that this principal might be a conjunction of named principals $n_1 \wedge \dots \wedge n_k$ representing a set of $k$ colluding principals. In the sections that follow, we denote a $\Nameset$-indexed set of memories as a function $\globalstore: \Nameset \to (\Loc \partialto \val)$.

\subsection{Trace semantics}
We express the attacker model in terms of a trace semantics, in which certain operations in the language emit \emph{events} which may or may not be observable by $\Attacker$. The grammar for events is given in Figure~\ref{fig:event-syntax}. A non-empty event $(\alpha, \env, n)$ contains the type of the event $\alpha$, the current environment when the event was emitted $\env$, and the node $n$ that emitted the event. The types of events include: write events $\writeevent{\addr}{\expr}$, emitted when a node writes an expression $\expr$ to reference $\addr$; allocation events $\newevent{\addr}{\expr}$, emitted when a node allocates a new reference $\addr$, initialized to $\expr$; call events $\callevent{\expr}{m}$, emitted when a node invokes an RPC $\expr$ on node $m$; and return events $\returnevent{\val}{m}$, emitted when a node finishes an RPC, returning value $\val$ to the caller node $m$. In addition, we have release events $\releaseeventkw$, explained below. Finally, the empty event $\noevent$ is emitted by operations that do not emit some other event. We call a sequence of events a \emph{trace}. We write the concatenation of traces as $t_1 \cdot t_2$ and we write the empty trace as $\emptytrace$.
%An attacker observes modifications to memory (i.e., writes or allocations) locally on a node and remote procedure calls and returns across nodes.

\begin{figure}
\centering
\begin{align*}
\event &\vcentcolon\vcentcolon= (\alpha, \env, n) \mid \noevent\\
\alpha &\vcentcolon\vcentcolon= \writeevent{\addr}{\expr} \mid \newevent{\addr}{\expr} \mid \callevent{\expr}{n} \\ &\quad \mid \returnevent{\val}{n} \mid \releaseevent{p}{q}{r}
\end{align*}
\caption{The syntax of events.}
\label{fig:event-syntax}
\end{figure}

\begin{figure}
\centering
\begin{mathpar}
\inferrule[E-Write-Ev]{[\ldots] \and \event = (\mathsf{write}(\addr, \expr'), \env, n)}{\step{n; \env}[][\event]{\config{\store}{\efill{\writeref{\addr}{\expr'}}}}{\config{\store'}{\efill{\return{()}}}}{\sigma}}
\and
\inferrule[G-Step-Ret-Ev]{S(n) = \config{\store_n}{\efill{\wait[\type]}} \and S(m) = \config{\store_m}{\val ; \overline{\expr_m}} \\\\ \event = (\mathsf{ret}(\val, n), \env, m)}{\gconfig{m \cdot n \cdot \overline{n}}{\env}{S} \gstepsto[][\event] \gconfig{n \cdot \overline{n}}{\env}{\extends{S}{n \mapsto s_n', m \mapsto s_m'}}}
\and
\inferrule[G-Step-Local-Ev]{\step{n; \env}[][\event]{S(n)}{s}{\sigma}}{\gconfig{n \cdot \overline{n}}{\env}{S} \gstepsto[][\event] \gconfig{n \cdot \overline{n}}{\extend{\env}{n}{\sigma}}{\extend{S}{n}{s}}}
\and
\inferrule[E-Assume-Ev]{
[\ldots] \and \event = (\mathsf{release}(p, q, r), \env, n)}{\step{n; \env}[][\event]{\config{\store}{\efill{\adddelegate{p}{q}{r}}}}{ \config{\store}{\efill{\return{()}}}}{\sigma}}
\end{mathpar}
\caption{Augmented semantics emitting events.}
\label{fig:event-semantics}
\end{figure}

\paragraph{Release events}\label{sec:release-events}
\lang{} is a very expressive language that permits downgrading, i.e., intentionally relaxing information-flow restrictions on data \cite{Myers:1999:JPM:292540.292561}. To define noninterference, we are concerned only with executions that do not downgrade information to $\Attacker$ (or, more precisely, from $p$ to $q$ where $p \nflowsto \Attacker$ and $q \flowsto \Attacker$).
We expect generalizations of noninterference, like robust declassification \cite{Zdancewic:2001:RD:872752.873524, Myers:2004:ERD:1009380.1009673} and nonmalleable information-flow \cite{Cecchetti:2017:NIF:3133956.3134054} to hold for \lang, but in this work we consider only the case where no node downgrades information to $\Attacker$. To capture this intuition, we introduce the notion of a \emph{bad} release event. Intuitively, when no bad release events are emitted nothing is being downgraded to $\Attacker$.
%To define noninterference, we require that nodes agree on which principals flow to $\Attacker$. That is, it must \emph{not} be the case that $\mathsf{Alice}$ considers data to not be observable by $\Attacker$, and then send it to $\mathsf{Bob}$, who considers the same data to be observable by $\Attacker$.

\paragraph{Bad release events}
We call downgrading of confidentiality labels \emph{declassification}, and downgrading of integrity labels \emph{endorsement}. As \lang{} permits both declassification and endorsement using delegations, a bad event should capture both cases. We call a release event $\event$ $=$ $(\mathsf{release}(p, q, r), \env, n)$ bad, written $\bad{\Attacker}{\event}$, if $\nopostflowstoquery{n; \env}{r}{\Attacker}$ and one of the following conditions hold:
\begin{enumerate}
    \item \label{enum:bad-event-conf} $\nopostflowstoquery{n; \env}{p}{\conf{\Attacker}}$ and $\nopostnotflowstoquery{n; \env}{q}{\conf{\Attacker}}$
    \item \label{enum:bad-event-integ} $\nopostflowstoquery{n; \env}{\integ{\Attacker}}{p}$ and $\nopostnotflowstoquery{n; \env}{\integ{\Attacker}}{q}$
\end{enumerate}
The condition $\nopostflowstoquery{n; \env}{r}{\Attacker}$ captures that a release event can be bad only if $\Attacker$ can observe the delegation. Condition~\ref{enum:bad-event-conf} captures bad declassifications: the new delegation gives $\Attacker$ the authority to observe values labeled as $q$ since $\Attacker$ can already observe values labeled as $p$. Similarly, condition \ref{enum:bad-event-integ} captures bad endorsements: the new delegation gives $\Attacker$ the authority to write values labeled as $q$ since $\Attacker$ can already write values labeled as $p$.
%Figure~\ref{fig:bad-release} illustrates a bad release event that declassifies information at $q$ to $\Attacker$: if $\conf{p} \flowsto \conf{\Attacker}$, the delegation $\lb{r}{p \actsfor q}$ can be used to prove that $\conf{q} \flowsto \conf{\Attacker}$ using transitively.

\paragraph{Examples of bad release events}
The release event $\event$ $=$ $(\releaseevent{\Attacker}{\conf{\mathsf{Alice}}}{\botinfoflow}, \env, n)$ generated by the local step
\begin{equation*}
\step{n; \env}[][\event]{\config{\store}{\adddelegate{\Attacker}{\conf{\mathsf{Alice}}}{\botinfoflow}}}{\config{\store}{\expr}}{\sigma}
\end{equation*}
(i.e., a declassification from $\conf{\mathsf{Alice}}$ to $\Attacker$) is bad since $\nopostflowstoquery{n; \env}{\botinfoflow}{\Attacker}$ (i.e., $\Attacker$ can observe the delegation), $\nopostflowstoquery{n; \env}{\Attacker}{\conf{\Attacker}}$ (i.e., it is a declassification to a principal that $\Attacker$ can observe), and $\nopostnotflowstoquery{n; \env}{\conf{\mathsf{Alice}}}{\conf{\Attacker}}$ (i.e., it is a declassification from a principal that $\Attacker$ previously could not observe). The release event $\event = (\releaseevent{\Attacker}{\integ{\mathsf{Bob}}}{\botinfoflow}, \env, n)$ generated by the local step
\begin{equation*}
\step{n; \env}[][\event]{\config{\store}{\adddelegate{\Attacker}{\integ{\mathsf{Bob}}}{\botinfoflow}}}{\config{\store}{()}}{\sigma}
\end{equation*}
(i.e., an endorsement from $\Attacker$ to $\integ{\mathsf{Bob}}$) is bad since $\nopostflowstoquery{n; \env}{\botinfoflow}{\Attacker}$ (i.e., $\Attacker$ can observe the delegation), $\nopostflowstoquery{n; \env}{\integ{\Attacker}}{\Attacker}$ (i.e., it is an endorsement from a principal that $\Attacker$ can modify), and $\nopostnotflowstoquery{n; \env}{\integ{\Attacker}}{\integ{\mathsf{Bob}}}$ (i.e., it is an endorsement to a principal that $\Attacker$ previously could not modify).

Finally, the release event $\event = (\releaseevent{\Attacker}{\mathsf{Charlie}}{\botinfoflow}, \env, n)$ is bad as it corresponds to both a declassification and an endorsement, as both condition 1 and condition 2 holds.

When a release event is not bad, we say that it is \emph{good}, and we extend the definition of good release events to traces: a trace $t$ is \emph{good}, written $\good{\Attacker}{t}$, if $t$ does not contain any bad release events. Our noninterference result, presented at the end of this section, is quantified over good traces only, and we leave the problem of extending this result to more relaxed notions of noninterference as future work.

\paragraph{$\Attacker$-equivalence}
Given a trace $t$ the $\Attacker$-observable trace of $t$ is the trace $t \upharpoonright \Attacker$, defined as
\begin{align*}
\emptytrace \upharpoonright \Attacker &= \emptytrace\\
((\alpha, \env, n) \cdot t) \upharpoonright \Attacker &=
\begin{cases}
(\alpha, \env, n) \cdot (t \upharpoonright \Attacker) & \nopostflowstoquery{n; \env}{\env(n).\lblkw}{\Attacker} \\
t \upharpoonright \Attacker & \text{otherwise}
\end{cases}
\end{align*}
We augment the semantics from Section~\ref{sec:calculus} with events. Figure~\ref{fig:event-semantics} shows an excerpt of the augmented semantics (we use $[\ldots]$ to elide the premises presented in Section~\ref{sec:calculus}). Except for the emitted event these rules correspond exactly to the rules in Figures~\ref{fig:monadic-reductions-1}, \ref{fig:monadic-reductions-2} and \ref{fig:global-steps}. We write $\gconfig{n}{\env}{S} \gstepstos[][t]$ when there exists a configuration $\gconfig{n'}{\env'}{S'}$ such that $\gconfig{n}{\env}{S} \gstepstos[][t] \gconfig{n'}{\env'}{S'}$ and $S'(n) = \config{\store_n}{\termsym}$ for all $n$. We also write $\gconfig{n}{\env}{S} \gstepstos[][\Attacker \filtertrace t']$ when $\gconfig{n}{\env}{S} \gstepstos[][t]$ and $t' = t \upharpoonright \Attacker$.

We define an $\Attacker$-equivalence relation that makes explicit which traces and memories an attacker $\Attacker$ can distinguish. As they both contain expressions, we define an $\Attacker$-equivalence on expressions, and Figure~\ref{fig:low-eq-expr} shows an excerpt of this judgment. Intuitively, two expressions $\expr_1$ and $\expr_2$ are considered $\Attacker$-equivalent if the current label on each context does not flow to $\Attacker$, or if the label on each context flows to $\Attacker$ and $\expr_1$ and $\expr_2$ are equal ``up to labeled values with a label that does not flow to $\Attacker$''. Figure~\ref{fig:low-eq-expr} formalizes this intuition: rule \ruleref{Eq-High} states that two expressions are $\Attacker$-equivalent in an environment with a current label that does not flow to the attacker, and \ruleref{Eq-Low} states that two expressions are $\Attacker$-equivalent if the label on each context flows to $\Attacker$, and the expressions are equal up to labeled values that $\Attacker$ cannot observe. This is expressed by the judgment $\loweql{C}{\Attacker}[\bijection]{\expr_1}{\expr_2}$.

For most cases $\loweql{C}{\Attacker}[\bijection]{\expr_1}{\expr_2}$ recursively inspects the subexpressions of each expression, but a few cases needs special case: to relate dynamically allocated locations we use a partial bijection \cite{Banerjee:2002:SIF:794201.795164, Rajani2018} $\bijection: \Loc \partialto \Loc$ in \ruleref{Eq-Addr}. The most important rules are \ruleref{Eq-Labeled-1} (that makes explicit the notion that an attacker cannot distinguish two terms labeled with a principal which does not flow to $\Attacker$), and \ruleref{Eq-Labeled-2} (that states that $\Attacker$ can ``look inside'' terms labeled with principals that flow to $\Attacker$). When $\bijection$ is not important we write $\loweq{n; \env_1; \env_2}{\Attacker}{\expr_1}{\expr_2}$ to mean $\loweq{n; \env_1; \env_2}{\Attacker}[\bijection]{\expr_1}{\expr_2}$ for some bijection $\bijection$.

\begin{figure}
    \centering
    \begin{mathpar}
    \inferrule[Eq-High]{\nopostnotflowstoquery{C_i}{\env_i(n).\lblkw}{\Attacker}, \\\\ i = 1,2}{\loweq{C}{\Attacker}[\bijection]{\expr_1}{\expr_2}}
    \and
    \inferrule[Eq-Low]{\nopostflowstoquery{C_i}{\env_i(n).\lblkw}{\Attacker}, i = 1,2 \\\\ \loweql{C}{\Attacker}[\bijection]{\expr_1}{\expr_2}}{\loweq{C}{\Attacker}{\expr_1}{\expr_2}}
    \and
    \inferrule[Eq-Addr]{\bijection(\addr_1) = \addr_2}{\loweql{C}{\Attacker}[\bijection]{\addr_1}{\addr_2}}
    \and
    \inferrule[Eq-Labeled-1]{\nopostnotflowstoquery{C_i}{p_i}{\Attacker}\quad i = 1,2}{\loweql{C}{\Attacker}[\bijection]{\lb{p_1}{\expr_1}}{\lb{p_2}{\expr_2}}}
    \and
    \inferrule[Eq-Labeled-2]{\nopostflowstoquery{C_i}{q}{\Attacker} \quad i = 1,2 \and \loweql{n; \env_1; \env_2}{\Attacker}[\bijection]{\expr_1}{\expr_2}}{\loweql{C}{\Attacker}[\bijection]{\lb{q}{\expr_1}}{\lb{q}{\expr_2}}}
    \end{mathpar}
    \caption{$\Attacker$-equivalence for terms and expressions. The meta-variable $C$ abbreviates $n; \env_1; \env_2$, and $C_i = n; \env_i$.}
    \label{fig:low-eq-expr}
\end{figure}

The $\Attacker$-equivalence relation on expressions induces an $\Attacker$-equivalence relation on events, and $\Attacker$-equivalence of traces are defined as pairwise $\Attacker$-equivalence of the events in the trace. We write $t_1 \simeq_{\Attacker}^{\bijection} t_2$ for $\Attacker$-equivalence on traces, and $t_1 \simeq_{\Attacker} t_2$ to mean $t_1 \simeq_{\Attacker}^{\bijection} t_2$ for some bijection $\bijection$.

Finally, Figure~\ref{fig:low-eq-memories} shows $\Attacker$-equivalence on memories. Rule \ruleref{Store-Eq-Empty} states that two empty memories are $\Attacker$-equivalent, and rule \ruleref{Store-Eq-Low} states that extending two memories with $\Attacker$-equivalent expressions preserves $\Attacker$-equivalence. Lastly, rules \ruleref{Store-Eq-High-1} and \ruleref{Store-Eq-High-2} states that both memories can be extended with terms labeled with a principal that does not flow to $\Attacker$ without the attacker being able to distinguish the memories. We extend the notion of $\Attacker$-equivalence on memories to $\Nameset$-indexed sets of memories and write $\loweq{m; \env_1; \env_2}{\Attacker}[\bijection]{\globalstore}{\altglobalstore}$ when $\forall n \in \Nameset.~\loweq{m; \env_1; \env_2}{\Attacker}[\bijection]{\globalstore(n)}{\altglobalstore(n)}$.

To simplify the statement of our end-to-end security guarantee note that, for any $n, m \in \Nameset$ we have $\loweq{n; \emptyenv; \emptyenv}{\Attacker}{\store}{\altstore}$ if and only if $\loweq{m; \emptyenv; \emptyenv}{\Attacker}{\store}{\altstore}$. Thus, we write $\store \simeq_{\Attacker} \altstore$ to mean $\loweq{n; \emptyenv; \emptyenv}{\Attacker}{\store}{\altstore}$ for some $n$.

\begin{figure}
    \centering
    \begin{mathpar}
    \inferrule[Store-Eq-Empty]{~}{\loweq{C}{\Attacker}[\bijection]{\varnothing}{\varnothing}}
    \and
    \inferrule[Store-Eq-Low]{\bijection(\addr_1) = \addr_2 \and \loweq{C}{\Attacker}[\bijection]{\expr_1}{\expr_2} \\\\ \nopostflowstoquery{C_i}{q}{\Attacker}\quad i = 1,2 \\\\ \loweq{C}{\Attacker}[\bijection]{\store_1}{\store_2} \\\\ \store_i' = \extend{\store_i}{\addr_i}{(\lb{q}{\expr_i})}}{\loweq{C}{\Attacker}[\bijection]{\store_1'}{\store_2'}}
    \and
    \inferrule[Store-Eq-High-1]{\nopostnotflowstoquery{C_1}{q}{\Attacker} \\\\ \loweq{C}{\Attacker}[\bijection]{\store_1}{\store_2}}{\loweq{C}{\Attacker}[\bijection]{\extend{\store_1}{\addr}{\lb{q}{\expr}}}{\store_2}}
    \and
    \inferrule[Store-Eq-High-2]{\nopostnotflowstoquery{C_2}{q}{\Attacker} \\\\ \loweq{C}{\Attacker}[\bijection]{\store_1}{\store_2}}{\loweq{C}{\Attacker}[\bijection]{\store_1}{\extend{\store_2}{\addr}{\lb{q}{\expr}}}}
    \end{mathpar}
    \caption{$\Attacker$-equivalence for memories. The meta-variable $C$ abbreviates $n; \env_1; \env_2$, and $C_i = n; \env_i$.}
    \label{fig:low-eq-memories}
\end{figure}

\paragraph{Attacker knowledge}
We present our noninterference result using the notion of attacker knowledge \cite{Askarov:2008:TNL:1462455.1462485, 4223226}. The attacker knowledge is the set of initial memories that could lead to a given observable trace, and a larger knowledge set corresponds to more uncertainty about the initial memory. Formally, attacker knowledge, given a trace $t$ produced by expression $\expr$, is the set $\knowledge{n}(\expr, t)$.
\begin{equation*}
\knowledge{n}(\expr, t) = \left\{ \, \altglobalstore \, \middle\vert \, \gconfig{n}{\emptyenv}{S} \gstepstos[][\Attacker \filtertrace t'] \wedge \, t \simeq_{\Attacker} t' \, \right\}
\end{equation*}
where $S(m) = \config{\altglobalstore(n)}{\llbracket \expr \rrbracket_n(m)}$ and
\begin{equation*}
\llbracket \expr \rrbracket_n(m) =
\begin{cases}
\expr & \text{ if } n = m\\
\termsym & \text{ otherwise }
\end{cases}
\end{equation*}
ensures that expression $\expr$ is initially evaluated on node $n$, and the remaining nodes start with an empty list of expressions.
As is standard for termination-insensitive noninterference (TINI), the policy \cite{6234468} is defined as all $\Attacker$-equivalent terminating memories, which we denote by $\termknowledge{n}(\globalstore, \expr)$.
\begin{equation*}
\termknowledge{n}(\globalstore, \expr) = \left\{ \,\altglobalstore \, \middle\vert \, \globalstore \simeq_{\Attacker} \altglobalstore \wedge \, \gconfig{n}{\emptyenv}{S} \gstepstos \, \right\}
\end{equation*}
where $S(m) = \config{\altglobalstore(n)}{\llbracket \expr \rrbracket_n(m)}$. TINI can now be stated for \lang{} as a guarantee that two traces, generated by two evaluations of a well-typed expression $\expr$ starting with $\Attacker$-equivalent memories $\globalstore$ and $\altglobalstore$, are $\Attacker$-equivalent. Using attacker knowledge, we can succinctly write this as the inclusion of $\termknowledge{n}(\globalstore, \expr)$ in $\knowledge{n}(\expr, t)$: all $\Attacker$-equivalent terminating memories generate $\Attacker$-observable traces.

\begin{theorem}[Noninterference]\label{thm:ni}
Let $\expr$ satisfy $\hastype{n; \tenv}{\expr}{\type}$.
If $\gconfig{n}{\varnothing}{S} \gstepstos[][\Attacker \filtertrace t]$ for $S(m) = \config{\globalstore(m)}{\llbracket \expr \rrbracket_n(m)}$ such that $\good{\Attacker}{t}$ then $\knowledge{n}(\expr, t) \supseteq \termknowledge{n}(\globalstore, \expr)$.
\end{theorem}

%Theorem~\ref{thm:ni} is proved using a generalized version of the statement that quantifies over any two $\Attacker$-equivalent global configurations $\gconfig{n_1}{\env_1}{S_1}$, $\gconfig{n_2}{\env_2}{S_2}$ that proves they emit $\Attacker$-equivalent observable traces. An important invariant obtained by quantifying only over good traces is that whenever $\actsforquery{n; \env_1}{p}{q}{\level}$ and $\nopostflowstoquery{n; \env_1}{\level}{\Attacker}$ it holds that $\actsforquery{n; \env_2}{p}{q}{\level}$ and $\nopostflowstoquery{n; \env_2}{\level}{\Attacker}$.
The technical report \cite{techreport} contains a complete proof of Theorem~\ref{thm:ni}. The theorem shows that we have securely integrated FLAM into an LIO-like setting with a floating label, where proofs of trust relationships do not inappropriately reveal confidential information, nor are they inappropriately affected by untrusted information.