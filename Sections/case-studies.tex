We have implemented \lang{} as a monadic library in Haskell \cite{flamiolib}. The code is approximately 2,100 lines of code in total, and uses FLAM's efficient query resolution algorithm for authorization queries~\cite{Arden:2015:FA:2859845.2859998}. Proof search for trust relationships is implemented as computations in the \lang{} monad, ensuring that delegations are not used inappropriately. The case studies demonstrate how application-specific search strategies are used to prevent label creep during proof search.

Along with the efficient query resolution algorithm, we cache query results to avoid repeated network communication.\footnote{We do not consider side channel attacks introduced by caching queries, such as external timing attacks, in this work.}
To simplify the implementation, we differ from the calculus in the following ways:
\begin{enumerate}
    \item \label{enum:no-func-across-network} An RPC does not send the function that should be called across the network. Instead, the receiver of the RPC has a table mapping identifiers to functions, and the caller sends this identifier along with the list of arguments.
    \item \label{enum:cache-network-queries} Since query results obtained via network communication are cached on a per-query basis, no two identical queries are sent to the same node.
\end{enumerate}

First, \ref{enum:no-func-across-network}) does not lead to loss of expressivity: as shown by Cooper and Wadler~\cite{Cooper:2009:RC:1599410.1599439},  a program in a calculus similar to that of Section~\ref{sec:calculus} can be translated by performing defunctionalization to a Haskell program (which can then use the \lang{} implementation). 

Second, \ref{enum:cache-network-queries}) significantly reduces network communication but means that the implementation is unsound if the trust relationship between principals changes during query resolution. Orthogonal work on query isolation \cite{Liu:2009:FPS:1629575.1629606} can provide transactional behavior for distributed systems like \lang.

Using this implementation, we have constructed three use cases for \lang{} consisting of roughly 500 lines of code. The first use case is a distributed bank, which was already presented in Section~\ref{sec:programming}. The banking example shows how users can perform remote procedure calls to handle transactions across accounts between different users. A user $u$ can authorize user $u'$ to transfer money on behalf of $u$ by adding a delegation $\lb{\conf{\mathsf{bank}} \conj \integ{u}}{u' \actsfor u}$ locally on $u$'s node. This delegation is read as ``$u$ trusts $u'$, and this information is confidential to $\mathsf{bank}$ and has the integrity of $u$''. When $\mathsf{bank}$ wishes to prove the trust relationship between $u$ and $u'$ to authorize a transfer of money, a proof search is issued, and using the \ruleref{Fwd} rule, is forwarded to node $u$. Node $u$ then proves the trust relationship using the local delegation $\lb{\conf{\mathsf{bank}} \conj \integ{u}}{u' \actsfor u}$.

In addition to the banking example, we construct a secure social jukebox service where people schedule music during social gatherings. The third use case is a secure database containing confidential information about government agencies. The third example also demonstrates how the \ruleref{Assump} rule prevents infinite derivation trees in authorization queries, and how such queries can show up in practical use cases.

\subsection{Secure social jukebox service}\label{subsec:jukebox}
Suppose a group of principals $\Nameset$ is gathered at a party and want to vote on which songs should be played at the party, but do not want their votes leaked to unauthorized principals, nor do they want unauthorized principals to vote on their behalf.
For instance, if $\mathsf{Alice}$ votes for ``Taylor Swift - Shake It Off'', she wants to ensure that only principals she trusts can learn her vote for this song. Furthermore, only principals that $\mathsf{Alice}$ trusts should be able to vote for a song on her behalf.

We assume a distinguished principal $J \in \Nameset$ (for jukebox) representing a node on which two functions exist:
\begin{align*}
\mathtt{get}&: \liotype{(\pair{\String}{\listtype{\labeled{\String}}})}\\
\mathtt{put}&: \pair{\String}{\listtype{\labeled{\String}}} \to \liotype{\unit}
\end{align*}
Function $\mathtt{get}$ returns a pair $(\mathit{s}, \mathit{lss})$ containing the current song being played $s$, and a list of labeled strings $\mathit{lss}$ such that $\lb{p}{s'} \in \mathit{lss}$ represents that $p$ voted for $s'$. So if $\mathsf{Alice}$ wants to vote for ``Taylor Swift - Shake It Off'', she appends a labeled string $\lb{\mathsf{Alice}}{\texttt{"Taylor Swift - Shake It Off"}}$ using $\mathtt{put}$. By labeling her vote with the principal $\mathsf{Alice}$, she knows that only principals $p$ such that $\conf{\mathsf{Alice}}$ flows to $\conf{p}$ can learn her vote (i.e., by \ruleref{E-Unlabel} it must be the case that $\mathsf{Alice}$ flows to $\conf{p}$). Furthermore, since the integrity of the label on the vote is $\integ{\mathsf{Alice}}$, she knows that any vote of the form $\lb{\mathsf{Alice}}{s}$ for some song $s$ must be placed by a principal $p$ such that $\integ{\mathsf{Alice}}$ trusts $\integ{p}$ (i.e., by \ruleref{E-Label} it must be the case that $p$'s current label flows to $\mathsf{Alice}$).

Figure~\ref{fig:jukebox-alice-votes-for-taylor-swift} shows an example of $\mathsf{Alice}$ voting for ``Shake it Off'' by Taylor Swift. First, $\mathsf{Alice}$ adds a delegation that allows $J$ to read Alice's labeled vote. She then labels her vote and inserts it into the list of labeled songs. When $J$ then unlabels the labeled song title, a proof search will be issued checking $\integ{J} \join \mathsf{Alice}$ flows to $\conf{J}$, which is equivalent to checking that $\conf{\mathsf{Alice}}$ trusts $\conf{\mathsf{Alice}} \conj \integ{(J \disj \mathsf{Alice})}$. This trust relationship holds by the delegation that Alice placed in Figure~\ref{fig:jukebox-alice-votes-for-taylor-swift}.

\begin{figure}
\centering
\begin{lstlisting}
((*@$\lambda^{\mathsf{Alice}}_{\integ{\mathsf{Alice}}}$@*) _ . do
  assume (*@$\conf{J}$@*) (*@$\actsfor$@*) (*@$\conf{\mathsf{Alice}}$@*) @ (*@($\conf{\bot} \conj \integ{\mathsf{Alice}}$)@*)
  ls <- label (*@$\mathsf{Alice}$@*) "Taylor Swift - Shake It Off"
  (curSong, votedSongs) <- get
  put (curSong, ls :: votedSongs)) ()
\end{lstlisting}
\caption{$\mathsf{Alice}$ places a secret vote for Taylor Swift.}
\label{fig:jukebox-alice-votes-for-taylor-swift}
\end{figure}

\subsection{Government agency records}\label{subsec:agents}
As a final example demonstrating the usefulness of \lang, we show how confidential delegations can be used to keep government agency records. Suppose the CIA hires a subset of $\Nameset$  as agents. 
This information should visible only to other CIA agents, not to the general public. Furthermore, only the CIA should be able to hire agents.

We implement the an enforcement mechanism for this security policy as follows: when $\mathsf{CIA} \in \Nameset$ hires an agent $n \in \Nameset$, $\mathsf{CIA}$, two delegations are created: first, trust is delegated from $\mathsf{CIA}$ to $\owner{\mathsf{CIA}}{n}$. By the properties of ownership projections \cite{Arden:2015:FA:2859845.2859998}, this delegation can be read as ``$\mathsf{CIA}$ trusts $n$, but $\mathsf{CIA}$ controls which principals can act for $n$''.\footnote{The use of ownership principals prevent \emph{delegation loopholes} \cite{Arden:2015:FA:2859845.2859998}.} To satisfy the security policy, this delegation should be visible only to other agents, and CIA should be able to trust that any such delegation could have been placed only by the CIA. This directly translates into the label $\conf{(\owner{\mathsf{CIA}}{\mathsf{AgentDB}})} \conj \integ{\mathsf{CIA}}$ on the delegation.

Second, any agent $n$ should be able to check if a given principal $m$ is an agent. To achieve this, $\conf{(\owner{\mathsf{CIA}}{\mathsf{AgentDB}})}$ delegates trust to $\conf{n}$, meaning that $n$ can learn information labeled as $\owner{\mathsf{CIA}}{\mathsf{AgentDB}}$. To satisfy the security policy, this delegation should also be labeled as $\conf{(\owner{\mathsf{CIA}}{\mathsf{AgentDB}})} \conj \integ{\mathsf{CIA}}$.

Figure~\ref{fig:govt-alice-checks-bob} shows how $\mathsf{Alice}$ can verify that $\mathsf{Bob}$ is also a secret agent. On line~4, $\mathsf{Alice}$ and $\mathsf{Bob}$ are both hired as secret agents using the function $\mathsf{hire}$ defined on lines~1-3. 

At some point, $\mathsf{Alice}$ meets $\mathsf{Bob}$ ``in the field'' and wants to verify his claim that he is a secret agent. In order for $\mathsf{Alice}$ to verify this claim she checks if the principal $\mathsf{CIA}$ delegates to $\owner{\mathsf{CIA}}{\mathsf{Bob}}$. Using the strategy $[\conf{\mathsf{Alice}} \conj \integ{\mathsf{CIA}}]$, Alice states that she is only interested in using delegations that have confidentiality at most $\mathsf{Alice}$, and any delegation must have been placed by a principal that $\mathsf{CIA}$ trusts.

For $\mathsf{Alice}$ to prove this trust relationship she uses the \ruleref{Fwd} rule and delegates the proof search to $\mathsf{CIA}$. By \ruleref{Fwd}, she must check that $\integ{\mathsf{Alice}}$ (i.e., the current label of $\mathsf{Alice}$) flows to $\conf{\mathsf{CIA}}$ (i.e., the clearance label of $\mathsf{CIA}$), which holds by \ruleref{Bot}. Applying \ruleref{Del}, $\mathsf{CIA}$ must now check that $\conf{(\owner{\mathsf{CIA}}{\mathsf{AgentDB}})}$ trusts $\conf{\mathsf{Alice}}$. Applying \ruleref{Del} again, this reduces to showing that $\conf{(\owner{\mathsf{CIA}}{\mathsf{AgentDB}})}$ trusts $\conf{\mathsf{Alice}}$ under the assumption that $\conf{(\owner{\mathsf{CIA}}{\mathsf{AgentDB}})}$ trusts $\conf{\mathsf{Alice}}$, and so the trust relationship is established using \ruleref{Assump}.

Notice that without the \ruleref{Assump} rule, we would not be able to prove that $\conf{(\owner{\mathsf{CIA}}{\mathsf{AgentDB}})}$ trusts $\conf{\mathsf{Alice}}$ (or, equivalently, that $\owner{\mathsf{CIA}}{\mathsf{AgentDB}}$ can learn information labeled as $\mathsf{Alice}$): we would keep applying \ruleref{Del} without a way of terminating the derivation. But, intuitively, this relationship should hold as we have a delegation of trust from $\conf{(\owner{\mathsf{CIA}}{\mathsf{AgentDB}})}$ to $\conf{\mathsf{Alice}}$.
%However, to use the delegation to prove the trust, the trust needs to hold already to prove that the label on the delegation flows to the current strategy principal.
Rule \ruleref{Assump}  allows the proof search to assume the delegation when proving that it is secure to use the delegation, effectively expressing that ``it is secure to use the delegation because the delegation says so,'' a form of checked endorsement \cite{Chong:2007:SWA:1294261.1294265, DBLP:journals/corr/abs-1107-5594}.

\begin{figure}
\centering
\begin{lstlisting}
let hire = (*@$\lambda^{\mathsf{CIA}}_{\integ{\mathsf{CIA}}}$@*) (*@$\mathsf{n}$@*) . do
  assume (*@$\owner{\mathsf{CIA}}{\mathsf{n}}$@*) (*@$\actsfor$@*) (*@$\mathsf{CIA}$@*) @ (*@$\conf{(\owner{\mathsf{CIA}}{\mathsf{AgentDB}})} \conj \integ{\mathsf{CIA}}$@*)
  assume (*@$\conf{\mathsf{n}}$@*) (*@$\actsfor$@*) (*@$\conf{(\owner{\mathsf{CIA}}{\mathsf{AgentDB}})}$@*) @ (*@$\conf{(\owner{\mathsf{CIA}}{\mathsf{AgentDB}})} \conj \integ{\mathsf{CIA}}$@*)
in ((*@$\lambda^{\mathsf{CIA}}_{\integ{\mathsf{CIA}}}$@*) _ . do hire (*@$\mathsf{Alice}$@*); hire (*@$\mathsf{Bob}$@*)) ()
[...]
((*@$\lambda^{\mathsf{Alice}}_{\integ{\mathsf{Alice}}}$@*) _ . withStrategy [(*@$\conf{\mathsf{Alice}} \conj \integ{\mathsf{CIA}}$@*)] (do
               isAgent <- CIA:Bob (*@$\actsfor$@*) CIA
               if isAgent then [...] else [...])) ()
\end{lstlisting}
\caption{$\mathsf{Alice}$ verifies that $\mathsf{Bob}$ is a secret agent for the $\mathsf{CIA}$.}
\label{fig:govt-alice-checks-bob}
\end{figure}